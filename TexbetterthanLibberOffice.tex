\input{include/preambule_article.tex}

\author{Филиппенко Павел, Сибгатуллин Булат}
\title{Верстка научных отчетов в \LaTeX{}}
\date{\today}

\begin{document}    
    %% сделать красивый титульник
    \maketitle
    \thispagestyle{empty}

    \newpage
    \tableofcontents{} %содержание
    \newpage

    \section{\LaTeX{} что это такое?}

    \subsection{Введение}

    Любую лабораторную работу или научную статью важно не только сделать правильно и грамотно, но и добиться того, чтобы она была презентабильной и хорошо выглядела. В этом нам поможет культовая среда для верстки \LaTeX{}.
    %% вставить эмблему латеха
    Для оформления лабораторных работ или иных документов многие используют Word или LibreOffice Writer. Однако обе эти программы имеют ряд недостатков и в вопросе верстки научного текста значительно уступают \LaTeX{}.

    Что такое \LaTeX{}?  
    Согласно Википедии: \LaTeX{} -- наиболее популярный набор макрорасширений (или макропакет) системы компьютерной вёрстки \TeX, который облегчает набор сложных документов. Пакет позволяет автоматизировать многие задачи набора текста и подготовки статей, включая набор текста на нескольких языках, нумерацию разделов и формул, перекрёстные ссылки, размещение иллюстраций и таблиц на странице, ведение библиографии и др.
    Возможно данное определение звучит несколько сложновато. Не забивая голову лишним, можете считать \LaTeX инструментом оформления научных отчетов. Более глубокое понимание придет с практикой.

    \begin{figure}[h!]
        \centering
        \includegraphics[width = \textwidth]{TexCode_example.png}
        \caption{}
        % \label{}
    \end{figure}

    \newpage

    \subsection{Преимущества \LaTeX}

    Чем же так хорош \LaTeX? В целом можно выделить 4 основных преимущества

    \begin{enumerate}
        \item Подгоняет документ ко всем типографским стандартам. Таким образом, можно полностю сконцентрироваться на содержании документа, а не на его оформлении.
        \item Автоматизация процессов: автоматическая нумерация разделов и формул, автоматическое составление оглавления и многое другое. Меньше рутины.
        \item Открытая среда. Обширное community. В интернете можно найти ответы на любые ваши вопросы, большое количество информации, а так же пакетов, расширений и фич.
        \item Кроссплатформенность.
    \end{enumerate}

    \subsection{Как оно работает???}

    В некотором смысле работа \LaTeX похожа на известные вам языки программирования. На вход подается исходный код, затем происходит компиляция и на выходе мы получаем тепленький красивый pdf-документ.

    %% картинки исходного кода и готового документа

    \section{Редакторы кода}

    Для работы с \LaTeX необходимо 2 вещи: специальный компилятор, который будет магическим образом превращать ваши исходники в pfd и редактор кода, в котором вы будете непосредственно работать.
    
    \begin{wrapfigure}{l}{0.25\textwidth}
        \centering
        \includegraphics[width = 0.2\textwidth]{texmaker220-220.jpg}
        \caption{}
        %\label{fig:}
    \end{wrapfigure}

    Существует большое количество программ, которые представляют собой визуальную графическую среду для создания и редактирования документов \LaTeX.
    Наиболее распространенным вариантом являются программы TexMaker и TexStudio. Это кросс-платформенные редакторы \LaTeX с открытым кодом. Данные программы являются интегрированными средами для создания \LaTeX документов и включают такие возможности, как интерактивная система проверки правописания, сворачивание блоков текста, подсветка синтаксиса и многое другое.

    \begin{wrapfigure}{r}{0.25\textwidth}
        \centering
        \includegraphics[width = 0.2\textwidth]{Texstudio_Logo.png}
        \caption{}
        %\label{fig:}
    \end{wrapfigure}

    % фотографи TexMaker и TexStudio

    %% Пара слов про VS-Code

    Мы же будем использовать веб-редактор латех документов OverLeaf. Его основное преимущество заключается в том, что данный редактор не требует никакой установки, воспользоваться им можно с любой платформы на любом устройстве с выходом в интернет.
    Кроме того, этот редактор позволяет нескольким пользователям редактировать один и тот же документ одновременно и просматривать изменения друг друга в режиме реального времени.

    \begin{figure}[h!]
        \centering
        \includegraphics[width=\linewidth]{OverLeaf_example.png}
        \caption{}
        \label{OverLeaf_example}
    \end{figure}

    \section{Overleaf}

    \subsection{Регистрация}

    Давайте же познакомимся с нашим текстовым редактором. Для этого перейдем на сайт 
    %%\href{https://www.overleaf.com/login}{https://www.overleaf.com/login} 
    https://www.overleaf.com/login и сразу попадем на страницу регистрации. 

    %% input a beautiful instruction pictures
    
    Чтобы зарегистрироваться, необходимо внизу экрана выбрать пункт Register.

    %% input a beautiful instruction pictures

    Вводим адрес электронной почты и придумываем пароль. После этого на вам предложат создать ваш первый проект, а на почту придет письмо с просьбой подтвердить адрес. Абсолютно ничего сложного,
    подтверждаем адрес и создаем наш первый проект Blank Project.

    %% input a beautiful instruction pictures

    \subsection{Знакомство с интерфейсом}

    После того, как вы создали ваш первый в жизни tex-project вас перенесет в рабочее пространство OverLeaf. Здесь будет
    проходить большая часть вашей работы.

    \begin{figure}[h!]
        \centering
        \includegraphics[width = \textwidth]{OverLeafWorkingSpace.png}
        \caption{}
        % \label{}
    \end{figure}

    Все пространство можно условно разделить на 3 главные смысловые области. 
    
    Первая область -- рабочая, здесь вы будете писать
    теховский код, который потом отправится на обработку компилятору. 
    
    Вторая область -- область визуализации, здесь вы
    можете посмотреть текущий вид вашего pdf-документа. 
    
    Третья область -- вспомогательная, в этой области вы можете видеть
    все файлы и папки вашего текущего проекта, а так же мини-оглавление вашей работы.

    %% вставить поясняющие картинки, цветом выделить обозначенные области

    Нажав на кнопочку <<Menu>> на верхней панели слева, вам откроется меню детальной настройки документа (что-то страшное, трогать пока не будем).
    А нажав на стрелочку рядом, мы выйдем в главное меню управления проектами.

    \begin{figure}[h!]
        \centering
        \includegraphics[width = \textwidth]{projects_manage_space.png}
        \caption{}
        % \label{}
    \end{figure}

    В этом меню вам доступно управление вашим аккаунтом, а так же управление всеми вашими латеховскими проектами проектами.

    \section{Начало работы}

    Наконец, мы приступаем к написанию нашего документа. Перейдем в наш <<проект Hello world!>> и в рабочей области удалим весь сгенерированный по умолчанию код.

    Перед тем, как наполнять наш документ содержанием и смыслом, необходимо настроить его вид и подключить используемые пакеты.

    %% временное решение для оформления исходноо кода
    %  окружение запрещает теху обрабатывать все, что находится внутри
    \begin{verbatim}
        \documentclass[a4paper,12pt]{article}

        % Русский язык
        \usepackage[T2A]{fontenc}
        \usepackage[utf8]{inputenc}	
        \usepackage[english,russian]{babel}

        % Математика
        \usepackage{amsmath,amsfonts,amssymb,amsthm,mathtools} 
    \end{verbatim}

    Первой командой мы требуем, чтобы \LaTeX{} сгенерировал наш документ типа article (статья) на бумаге размера A4, со стандартным шрифтом размера 12.
    Следующие 3 строчки подключают пакеты для работы с русским языком, а самая последняя строчка подключает математические пакеты.
    Кстати, как вы могли заметить, значком \% обозначаются комментарии. Комментарии помогут вам легче ориентироваться в вашем коде,
    все что начинается со значка \% будет игнорироваться компилятором.
    
    Итак, мы подключили базовые пакеты, без которых невозможна работа в латех. Далее мы подключим еще
    несколько пакетов, которые понадобятся нам в дальнейшем, а затем перейдем к написанию статьи.

    \begin{verbatim}
        \usepackage{graphicx}  % импорт изображений
        \graphicspath{images/} % папка с картинками
        \usepackage{caption}

        % центрирование подписи к картинке
        \captionsetup{justification=centering}

        \usepackage{hyperref}  % работа с сылками

        % настроим поля и колонтинулы страницы
        \usepackage{geometry} %
            \geometry{top=25mm}
            \geometry{bottom=35mm}
            \geometry{left=35mm}
            \geometry{right=20mm}
    \end{verbatim}

    Подготовительная работа закончена и мы можем приступить к написанию документа.

    %Добавить раздел работа с текстом, в нем описать нужные термины

    \section{Математика}

    Теперь, когда вы умеете работать с текстом вам нужно понять, как в \LaTeX{} записываются и оформляются математический выражения. Для выделения формул в \LaTeX{} используются две конструкции. Если получившиеся выражение вам важно и вы хотите его отметить (чтобы использовать в дальнейшем или по другой причине), то вам нужно использовать такую запись:
    
    \begin{equation}
        y = f(x)
    \end{equation}
    
    Вот так будет выглядеть ее код:    
    
    \begin{verbatim}
        \begin{equation}
            y = f(x)
        \end{equation}
    \end{verbatim}   
    
    Заметьте, что при такой форме записи \LaTeX{} самостоятельно записывает номер данного выражения, а следовательно, если вы хотите обратиться к какой-то формуле, уже записанной у вас в документе, вы можете просто записать её номер.
    
    В случае если ваша выражение является результатом промежуточных вычислений и не будет использоваться в дальнейшем, то стоит показать это при помощи второго вида записи:
    
    $$y = f(x)$$    
    
    И сделать это можно при помощи \$\$:    
    
    \begin{verbatim}
        $$y = f(x)$$
    \end{verbatim}
    
    Также, если вы хотите использовать в тексте какое-либо математическое выражение вам нужно выделить его \$ (теперь только одним):

    Some text: $e^x$. И вот так это выглядит в коде:
    
    \begin{verbatim}
        Some text: $e^x}
    \end{verbatim}
    
    Теперь посмотрим, как в \LaTeX{} будут записываться некоторые математические символы:
    
    $$e^x, e^{abc}$$
        
    $$e_x, e_{abc}$$
    
    $$e^{xyz}_{abc}$$
    
    \begin{verbatim}
        $$e^x, e^{abc}$$
        
        $$e_x, e_{abc}$$
        
        $$e^{xyz}_{abc}$$
    \end{verbatim} 
    
    Здесь вы можете видеть, как в \LaTeX{} записывается степень и индекс, а также из комбинация. Также обратите внимание на то, что если в степени или индексе больше одного символа, то их нужно брать в фигурные скобки.
    
    Часто бывает нужно (при использовании знака суммы или интеграла), чтобы степень и индекс перешли в пределы (например суммирования или интегрирования). Сделать это мы можем при помощи выражения $\backslash$limits, записанного после нужного нам символа, но перед его степенью и индексом:
    
    $$\int\limits^{x_{\text{макс}}}_{0}$$
    
    \begin{verbatim}
        $$\int\limits^{x_{\text{макс}}}_{0}$$
    \end{verbatim}
    
    Обратите внимание, что здесь мы использовали русский текст в записи выражения. Для его использования вам нужно будет написать $\backslash text{}$ и текст в фигурных скобках. При использовании английского языка можно просто писать на нем без использования $\backslash text{}$.
    
    Стоит отметить, что в \LaTeX{} часто приходится использовать формулы раной "высоты", поэтому здесь есть выражения для подгона высоты скобок под высоту формулы. Посмотрите как она реализована на примере обычной дроби:
    
    $$\left( \frac{x}{y} \right) $$
    
    \begin{verbatim}
        $$\left( \frac{x}{y} \right) $$
    \end{verbatim}
    
    Дробь здесь была задана выражением $\backslash$frac{}{} (в первой скобке записывается числитель, а во второй знаменатель. За подгон высоты скобок к высоте выражения отвечают надписи $\backslash$right и $\backslash$left перед скобками. Их вы можете использовать с люьым видом скобок (фигурными, прямоугольными и т.д.).
    
    
    
    На этом блок посвященный математике подходит к концу, и еси вы ожидали увидеть тут разбор существующи в \LaTeX{} символов, то мне стоит вас огорчить, так как этого тут не будет. Использование символов сводится к простому поиску кода символа и не представляетс из себя ничего сложного. Сложности обычно вызывают использование символов одновременно со степенями, дробями (в первую очередь из-за скобок), а это мы уже с вами разобрали.
    
    \section{Рисунки и таблицы}
    \section{Ссылки}
\end{document}